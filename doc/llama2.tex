\documentclass[letterpaper,12pt]{article}
\usepackage{times}
\usepackage{helvet}
\usepackage{courier}
\usepackage{fancyheadings}
\usepackage{hyperref}
\usepackage{xcolor}
\pagestyle{fancy}
\usepackage{graphicx}
\usepackage{verbatim}
\setlength\textwidth{6.5in}
\setlength\textheight{8.5in}
\newcommand{\TBC}{\framebox{\textbf{TO BE COMPLETED}}}
\newtheorem{assumption}{Assumption}
\newcommand{\be}{\begin{enumerate}}
\newcommand{\ee}{\end{enumerate}}
\newcommand{\bi}{\begin{itemize}}
\newcommand{\ei}{\end{itemize}}
\newcommand{\bd}{\begin{description}}
\newcommand{\ed}{\end{description}}
\newtheorem{notation}{Notation}
\begin{document}
\title{Karpathy Inference for Llama-2}
\author{Ramesh Subramonian }
\maketitle
\thispagestyle{fancy}
\lhead{}
\chead{}
\rhead{}
% \lfoot{{\small Decision Sciences Team}}
\cfoot{}
\rfoot{{\small \thepage}}

\begin{abstract}

\ Step-by-step re-implementation of Karpathy code for llama2.

\end{abstract}

\section{XXX}
\subsection{Conventions}

If \(T\) is a table of dimensions \(n_1 \times n_2\),
then \(T_i\) is a vector of length \(n_2\)

\subsection{Glossary}

  \begin{table}[hbtp]
    \centering
    \begin{tabular}{|l||l|l|l|} \hline \hline
      {\bf Abbreviation} & {\bf C code} &  {\bf Comments} \\ \hline
      \(n_D\) & {\tt dim} & \\ \hline
      \(n_{HD}\) & {\tt hidden\_dim} & \\ \hline
      \(n_L\) & {\tt n\_layers} & \\ \hline
      \(n_H\) & {\tt n\_heads} & \\ \hline
      \(n_{HKV}\) & {\tt n\_kv\_heads} & \\ \hline

      \(n_V\) & {\tt vocab\_size} & \\ \hline
      \(n_S\) & {\tt seq\_len} & \\ \hline
      \(s_H\) & {\tt head\_size} & \\ \hline
      \hline
      \(n_D'\) & {\tt ispc\_dim} & \\ \hline
      \hline
    \end{tabular}
      \caption{Scalars: Mapping math notation to C code} 
      \label{sclr_math_notation}
  \end{table}
%       \(S_{kc}\) & key\_cache &  \\ \hline 
%       \(S_{vc}\) & value\_cache & \\ \hline 

\begin{table}[hbtp]
\centering
\begin{tabular}{|l||l|l|l|} \hline \hline
{\bf Abbreviation} & {\bf C code} & {\bf Dimensions}  & {\bf Comments} \\ \hline 
\hline
\(W_t\) & token embedding table & 2 & \(n_V \times n_D\) \\ \hline
\(W_{att}\) & rms\_att\_weight & 2 & \(n_L \times n_D \) \\ \hline
\(W_{ffn}\) & rms\_ffn\_weight & 2 & \(n_L \times n_D \) \\ \hline
\(W_q\) & w\_q & 3 & \(n_L \times n_D \times (n_H \times s_H)\) \\ \hline
\(W_k\) & w\_k & 3 & \(n_L \times n_D \times (n_{HKV} \times s_H)\) \\ \hline
\(W_v\) & w\_v & 3 & \(n_L \times n_D \times (n_{HKV} \times s_H)\) \\ \hline
\(W_o\) & w\_o & 3 & \(n_L \times (n_H \times s_H) \times n_D\) \\ \hline
\hline
\end{tabular}
\caption{Weights: Mapping math notation to C code} 
\label{weights_math_notation}
\end{table}

\clearpage
\section{Utilities}

  \clearpage
  \section{Forward}
Arguments in Table~\ref{args_forward}
  \begin{table}[hbtp]
    \centering
    \begin{tabular}{|l||l|l|} \hline \hline
      {\bf Argument} & {\bf Type} & {\bf Comments}  \\ \hline 
      \(T\) & {\tt Transformer} & pointer \\ \hline 
      \(t\) & integer & token \(0 \leq t < n_V\) \\ \hline
      \(p\) & integer & pos \\ \hline
    \hline
    \end{tabular}
      \caption{Arguments for forward}
      \label{args_forward}
  \end{table}

\begin{figure}
\centering
\fbox{
\begin{minipage}{15cm}
  \begin{tabbing} % 
    \hspace*{0.25in} \= \hspace*{0.25in} \= \hspace*{0.25in} \= \kill

\(x \leftarrow W_t[t]\) \\ \\
    {\bf for}  each layer \(l\) {\bf do} \+ \\
    attention rmsnorm \\
    \(x_b \leftarrow {\mathrm rmsnorm}(x, W_{ra}[l]\) \\ 
    locate key and value in cache  \\
    \(x_{kc} \leftarrow (S_{kc}[l][p]\) \\ 
    \(x_{vc} \leftarrow (S_{vc}[l][p]\) \\ 
    \(XXX \leftarrow {\mathrm matmul}(x_{kc}, x_b, W_{wk}[l], )\)
SOME JUNK \- \\
    {\bf endfor} \\
  \end{tabbing}
\end{minipage}
}
      \caption{Pseudo-code for forward}
      \label{code_forward}
  \end{figure}

  \clearpage
\subsection{rmsnorm}
\label{rmsnorm}

Input is in Table~\ref{args_rmsnorm}. Output is \(o\), {\tt float} vector of length \(n\) 
  \begin{table}[hbtp]
    \centering
    \begin{tabular}{|l||l|l|} \hline \hline
      {\bf Argument} & {\bf Type}  \\ \hline 
      \(x\) & {\tt float} vector of length \(n\) \\ \hline
      \(w\) & {\tt float} vector of length \(n\) \\ \hline
    \hline
    \end{tabular}
      \caption{Arguments for rmsnorm}
      \label{args_rmsnorm}
  \end{table}

  \be
\item \(\alpha = ((\sum_i {x_i}^2)/n)+\epsilon\)
\item \(o_i \leftarrow \frac{ w_i \times x_i}{\alpha}\)
  \ee

  \clearpage
\subsection{softmax}
\label{softmax}

Arguments in Table~\ref{args_softmax}
  \begin{table}[hbtp]
    \centering
    \begin{tabular}{|l||l|l|} \hline \hline
      {\bf Argument} & {\bf Type}  \\ \hline 
      \(x\) & {\tt float} vector of length \(n\) \\ \hline
      \(n\) & integer \\ \hline
    \hline
    \end{tabular}
      \caption{Arguments for softmax}

      \label{args_softmax}
  \end{table}

  \be
\item \(m = {\mathrm max}_{i=0}^{i=n-1} x_i\)
\item \(\forall_{i=0}^{i=n-1} x_i \leftarrow e^{x_i - m}\)
\item \(s = \sum_{i=0}^{i=n-1} x_i\)
\item \(\forall_{i=0}^{i=n-1} x_i \leftarrow \frac{x_i}{s}\)
  \ee

\end{document}



